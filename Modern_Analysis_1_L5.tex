% Created 2023-09-06 Wed 15:37
% Intended LaTeX compiler: pdflatex
\documentclass[11pt]{article}
\usepackage[utf8]{inputenc}
\usepackage[T1]{fontenc}
\usepackage{graphicx}
\usepackage{longtable}
\usepackage{wrapfig}
\usepackage{rotating}
\usepackage[normalem]{ulem}
\usepackage{amsmath}
\usepackage{amssymb}
\usepackage{capt-of}
\usepackage{hyperref}
\author{Jim Liu}
\date{\today}
\title{Modern Analysis 1 L5}
\hypersetup{
 pdfauthor={Jim Liu},
 pdftitle={Modern Analysis 1 L5},
 pdfkeywords={},
 pdfsubject={},
 pdfcreator={Emacs 28.2 (Org mode 9.6.1)}, 
 pdflang={English}}
\begin{document}

\maketitle
\tableofcontents

\section{Superposition of Supremum}
\label{sec:orgd682576}
\begin{itemize}
\item THM: If \(A, B \subseteq \mathbb{R}\) are nonempty and bounded above, then \(A + B := \{a + b : a\in A, b\in B\}\) is bounded above and \(sup(A + B) = sup(A) + sup(B)\).
\begin{itemize}
\item Proof:\\[0pt]
Let \(\alpha = sup(A), \beta = sup(B)\)\\[0pt]
For each \(a \in A\) and \(b \in B\), we have \(a \leq \alpha, b \leq \beta\) \\[0pt]
So, \(a+b \leq \alpha + \beta\) (\(\mathbb{E}\))\\[0pt]
Therefore, \(a+b\) is an upper bound for \(A + B\).\\[0pt]
Therefore, \(sup(A+B) \leq sup(A)+sup(B)\) coz \(sup(A) + sup(B)\) is ``an'' upper bound, and \(sup(A+B)\) is the least upper bound.\\[0pt]
To prove the reverse equality: two approaches.\\[0pt]
\begin{itemize}
\item Option 1: show \(\alpha + \beta \leq u\) for each upper bound \(u\) of \(A + B\)\\[0pt]
Let u be any upper bound for \(A + B\)\\[0pt]
Fix \(a\in A\) and let \(b \in B\) be arbitrary,\\[0pt]
Then, \(a + b \leq u\)\\[0pt]
\(\therefore b \leq u-a\) for any \(b \in B\)\\[0pt]
\(\therefore u - a\) is an upper bound for B \(\Rightarrow \beta \leq u - a\)\\[0pt]
\(\therefore a \leq u - \beta\)\\[0pt]
Now, unfix \(a\), then \(u-\beta\) is an upper bound for A.\\[0pt]
So, \(\alpha \leq u - \beta\)\\[0pt]
\(\therefore \alpha + \beta \leq u\)\\[0pt]
\(\therefore \alpha + \beta = sup(A+B)\).
\item Option 2:\\[0pt]
Spse \(sup(A+B) < \alpha + \beta\) and then aim at \(\Rightarrow\!\Leftarrow\)\\[0pt]
Write \(\epsilon = \alpha + \beta - sup(A+B)>0\)\\[0pt]
Then, \(\alpha - \frac{1}{2} \epsilon < \alpha \Rightarrow \alpha - \frac{1}{2}\) is not an upper bound for A coz \(\alpha = lub(A)\) \\[0pt]
Thus, \(\exists a\in A: \alpha - \frac{1}{2}\epsilon < a\)\\[0pt]
Similarly, \(\exists b \in B : \beta - \frac{1}{2}\epsilon < b\)\\[0pt]
But now, \(a + b > \alpha + \beta - \epsilon = sup(A+B)\)\\[0pt]
\(\Rightarrow\!\Leftarrow\)
\end{itemize}
\item \(\mathbb{E}\) \(AB = \{ab : a\in A, b \in B\}\), \(sup(AB)=sup(A)sup(B)\). This works if \(A, B \subseteq \mathbb{R}^+\) and A, B are nonempty and bounded above.
\end{itemize}
\end{itemize}
\section{Square Root}
\label{sec:org3ed687e}
\begin{itemize}
\item THM: Each positive real numebr has a unique positive square root.
\begin{itemize}
\item Proof: \\[0pt]
Let \(a > 0\)\\[0pt]
\(\exists ! \alpha \in \mathbb{R}^+\) s.t. \(\alpha\alpha = a\)\\[0pt]
\begin{enumerate}
\item Uniqueness: \\[0pt]
\begin{itemize}
\item Option 1: \\[0pt]
Spse \(\alpha\alpha = a = \alpha' \alpha'\) (\(\alpha, \alpha' \in \mathbb{R}^+\))\\[0pt]
Then \(0 = \alpha'\alpha' - \alpha\alpha = (\alpha'+\alpha)(\alpha' - \alpha)\)\\[0pt]
Since \(\alpha' + \alpha > 0\), \(\alpha' - \alpha = 0\) \\[0pt]
So, \(0 = \alpha' - \alpha \Rightarrow \alpha' = \alpha\).
Hence, uniqueness is proved.
\item Option 2: \\[0pt]
Spse \(\alpha' \neq \alpha\) and then use trichotomy and aim for \(\Rightarrow\!\Leftarrow\)\\[0pt]
The above assumption leads to either \(\alpha' < \alpha \bigvee \alpha' > \alpha\). This implies that \(\alpha' \alpha' < \alpha\alpha\) \(\bigvee\) \(\alpha' \alpha' > \alpha\alpha\), both of which lead to \(\Rightarrow\!\Leftarrow\).
\end{itemize}
\item Existence:\\[0pt]
Define \(A = \{x\in \mathbb{R}^+: x^2<a \}\)\\[0pt]
Then, \(A \neq \varnothing\) and bounded above. Here, we need to show \textbf{nonempty} and \textbf{Bounded above}.
\begin{itemize}
\item Nonempty:\\[0pt]
If \(a<1, x=a\) coz \(a^2<a\)\\[0pt]
If \(a = 1, x= \frac{1}{2}a\)\\[0pt]
If \(a > 1, x = 1\).\\[0pt]
Therefore, A contains any real x satisfying \(0 < x < a \wedge 1\) (Here, \(\wedge\) means take the minimum value.)\\[0pt]
\(x < a \wedge 1 \Rightarrow x^{2}<x < a\)
\item Bounded above by \(a \vee 1\) (Here, \(\vee\) means take the maximum value.)\\[0pt]
Coming Next Lecture\ldots{}
\end{itemize}
\end{enumerate}
\end{itemize}
\end{itemize}
\end{document}